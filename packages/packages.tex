\usepackage[table]{xcolor}
\usepackage{pdfpages}
%\usepackage{cite}
\usepackage{tocloft}
%\usepackage{subfig}
\renewcommand{\cftfigfont}{Figura }
\renewcommand{\cfttabfont}{Tabla }
\usepackage{mathptm}
\date{\today}
\usepackage{braket}
\usepackage{multirow}
\usepackage{setspace}
\usepackage{verbatim}
\usepackage[a4paper]{geometry}
\geometry{top=3.0cm, bottom=2.5cm, left=3.0cm, right=2.5cm}
\usepackage{graphicx}
%\usepackage{subfigure}
\usepackage{caption}
\usepackage{subcaption}
\usepackage{amssymb}
\usepackage{float}
\usepackage{lmodern}
\usepackage[spanish,es-lcroman,es-tabla]{babel}
\usepackage[utf8]{inputenc}
\usepackage[T1]{fontenc}
\usepackage{textcomp}
\usepackage{appendix}
\usepackage{makeidx}
\usepackage{titlesec}
\usepackage{amsmath}
\usepackage[backend=biber,citestyle=ieee]{biblatex}
%\bibliography{contenido/Bibliografia/referencias}
\addbibresource{contenido/Bibliografia/referencias.bib}
\usepackage{fancyhdr}
% encabezado paginas normales solo BOOK
% [] es para pares {} es para impares
\lhead[]{} % izquierda
\chead[]{} % centro
\rhead[]{} % derecha
\renewcommand{\headrulewidth}{0pt} % grosor linea de separacion
% pie de pagina paginas normales solo BOOK
\lfoot[]{} % izquierda
\cfoot[\thepage]{\thepage} % centro - pone el numero de la pagina
\rfoot[]{} % derecha
\renewcommand{\footrulewidth}{0pt} % grosor linea de separacion
% encabezado y pie de pagina de los inicio de capitulo
\pagestyle{plain}{
	\fancyhead[L]{} % cabecera lado izquierdo
	\fancyhead[C]{} % cabecera lado centro
	\fancyhead[R]{} % cabecera lado derecha
	\fancyfoot[L]{} % pie pagina lado izquierdo
	\fancyfoot[C]{\thepage} % pie lado centro - pone numero de la pagina
	\fancyfoot[R]{} % pie pagina lado derecho
	\renewcommand{\headrulewidth}{0pt} % cabecera - grosor linea de separacion
	\renewcommand{\footrulewidth}{0pt} % pie pagina - grosor linea de separacion
}
    % opciones del paquete setspace
                         %\doublespacing
                         %\onehalfspace
                         %\singlespace
                         %\spacing{1.5}
                         %===========
\setlength{\parskip}{1cm plus 5mm minus 4mm}  %distancia entre parrafos


\newtheorem{hk}{Teorema}


\usepackage{array}
\definecolor{Gray}{gray}{0.85}
\definecolor{Task}{RGB}{184,230,168}
\definecolor{NoTask}{RGB}{224,224,224}
\definecolor{Cabecera}{RGB}{217, 234, 247}
\newcolumntype{a}{>{\columncolor{Gray}}l}

\renewcommand{\baselinestretch}{1.5}

\usepackage{svg}
\usepackage{hyperref}
\hypersetup{
    colorlinks=true,
    linkcolor=black,     
    urlcolor=black,
    citecolor=black,
    pdftitle={Control de estados electrónicos de defectos vía polarización ferroeléctrica en materiales luminiscentes bidimensionales},
    pdfpagemode=FullScreen,
    pdfauthor={Shigueru Nagata},
    }

\usepackage{pgfgantt}
\usepackage{booktabs}
\usepackage{multirow}
% comentario ramdon
